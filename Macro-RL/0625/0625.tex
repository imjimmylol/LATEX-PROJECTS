\documentclass{beamer}
\usepackage{hyperref}
\usepackage{adjustbox}
\usepackage{algorithm}
\usepackage{float}
\floatplacement{algorithm}{H}
\usepackage{algorithmicx}
\usepackage{etoolbox}        % for itemsep tweak (optional)
\usepackage{algcompatible}
\usepackage{algpseudocode}
\usepackage[CJKmath=true, AutoFakeBold = true]{xeCJK}
\usepackage[T1]{fontenc}
\setCJKmainfont{AR PL KaitiM GB} %中文字体设置为楷体。
% other packages
\usepackage{latexsym,xcolor,multicol,booktabs,calligra}
\usepackage{amssymb,amsfonts,amsmath,amsthm,mathrsfs,mathptmx} %数学常用宏包
\usepackage{graphicx,pstricks,listings,stackengine}
\usefonttheme[onlymath]{serif} %使用衬线字体展示公式,更加美观
\usepackage{soul}
\usepackage{tikz}
\usetikzlibrary{arrows.meta, positioning}
\usepackage[normalem]{ulem}
\renewcommand{\today}{\number\year 年\number\month 月\number\day 日}
\renewcommand{\alert}[1]{\textbf{\color{swufe}#1}}
\usepackage[CJKmath=true, AutoFakeBold = true]{xeCJK}
\usepackage[T1]{fontenc}
\setCJKmainfont{AR PL KaitiM GB} % 中文字体设置为楷体。
\usepackage[CJKmath=true, AutoFakeBold=3, AutoFakeSlant=.2]{xeCJK}
\setCJKmainfont{AR PL KaitiM Big5}  
\setCJKsansfont{AR PL KaitiM Big5}  
\setCJKmonofont{AR PL KaitiM Big5}
\newCJKfontfamily\Kai{標楷體}       
\XeTeXlinebreaklocale "zh"

\author{陳景龢}
\title{Paper Detail}
\subtitle{Deep Reinforcement Learning in a Monetary Model}
\institute[NCCU]
{\normalsize{Econ Dept\quad NCCU\quad }}
% \date{xxxx年xx月xx日}
\date{2025年6月3日} %直接使用编译当天日期
\usepackage{JNU}


% defs
\def\cmd#1{\texttt{\color[RGB]{0, 0, 139}\footnotesize $\backslash$#1}}
\def\env#1{\texttt{\color[RGB]{0, 0, 139}\footnotesize #1}}



\lstset{
    language=[LaTeX]TeX,
    basicstyle=\ttfamily\footnotesize,
    keywordstyle=\bfseries\color[RGB]{0, 0, 139},
    stringstyle=\color[RGB]{50, 50, 50},
    numbers=left,
    numberstyle=\small\color{gray},
    rulesepcolor=\color{red!20!green!20!blue!20},
    frame=shadowbox,
}

% \AtBeginSection[]{} % 取消章节切换前的目录
\AtBeginEnvironment{algorithmic}{\setlength{\itemsep}{2pt}}

\floatplacement{algorithm}{H}


\begin{document}


\begin{frame}
    \titlepage
    
    \begin{figure}[htpb]
        \begin{center}
            % \includegraphics[width=0.55\linewidth]{pic/jiangnan_logo.png}
        \end{center}
    \end{figure}
\end{frame}

\begin{frame}
    \tableofcontents[sectionstyle=show,subsectionstyle=show/shaded/hide,subsubsectionstyle=show/shaded/hide]
\end{frame}

\section{RL Framework}
    \begin{frame}{RL Components}
    The theoryotic frame work of RL is based on the \textbf{Markov Decision Process}
    
    \begin{enumerate}

        \item \textbf{Markov Decision Process : } \\
        $\bigl(\mathcal{S},\,\mathcal{A},\,P,\,R,\,\gamma\bigr)$
        
        \item \textbf{State:} \\
        $s_t \in \mathcal{S}, \quad \mathcal{S}: \text{State space}$
    
        \item \textbf{Action:} \\
        $a_t \in \mathcal{A}, 
        \quad \mathcal{A}: \text{Action space}$
    
        \item \textbf{Transition:} \\
        $P(s' \mid s, a)
        = \Pr\bigl(s_{t+1}=s' \mid s_t=s,\;a_t=a\bigr)$
    \end{enumerate}
\end{frame}

    \begin{frame}{Con'd}
        \begin{enumerate}
            \setcounter{enumi}{4}
            \item \textbf{Immediate Reward}: \\
            $R_{t+1}(s, a) = \mathbb{E}\bigl[R_{t+1} \mid s_t=s,\;a_t=a\bigr]$
            \item \textbf{Reward Function}: \\
            $G_t
            = R_{t+1} + \gamma\,R_{t+2} + \gamma^2\,R_{t+3} + \cdots
            = \sum_{k=0}^{\infty}\gamma^k\,R_{t+k+1}$

            \item \textbf{Policy}: \\
            $\pi(a \mid s)
            = \Pr\bigl(a_t=a \mid s_t=s\bigr)$

            \item \textbf{Action Value Funciton}: \\
            $Q^{\pi}(s,a)= \mathbb{E}_{\pi}\!\Bigl[G_t \mid s_t = s,\;a_t=a\Bigr]$


            \item \textbf{State Value Function}: \\
            $V^{\pi}(s)= \mathbb{E}_{a\sim\pi(.|s_t)}\!\Bigl[G_t \mid s_t = s\Bigr]$
            
        \end{enumerate}

    \end{frame}

    \begin{frame}{How Agent Choose actions}
        With the action value function and state value function, the agent can choose its action. For example : 
        \begin{enumerate}
            \item if we have optimal action value function\[
            \pi^*(s) = \arg\max_{a \in \mathcal{A}} Q^*(s, a)
            \]
            \item if we have optimal state value function \footnote{By applying the trick of writing $Q^*$ in terms of $V^*$}
            \[
            \pi^*(s) = \arg\max_{a \in \mathcal{A}}
            \Bigl\{\,R(s, a) + \gamma \sum_{s' \in \mathcal{S}} P(s' \mid s, a)\,V^*(s')\Bigr\}
            \]
        \end{enumerate}
    \end{frame}

    \begin{frame}{Categorizing RL algorithms}
        So, our goal now is to find the $Q^*$ and $V^*$. In order to find them, people developed different algorithms accordingly. Thus, in this slide, I use a popular way - learning goal (i.e weather agent learning $Q^*$ or $V^*$)
        to distinguish them 
        \begin{enumerate}
            \item Value learning : Learning $Q^*$
            \item Policy learning : Learning $V^*$
            \item Actor Critic : Learning both $V^*$ and $Q^*$ \footnote{參考DL 講義 Introduction to RL p.41}
        \end{enumerate}
    \end{frame}
    

\section{Actor Critic}

    \begin{frame}{Starting From State-Value Function}
        \begin{itemize}
            \item State-Value Function 
            \begin{align*}
            V^{\pi}(s) 
              &= \mathbb{E}_{a\sim\pi(.|s_t)}\bigl[G_t \mid s_t = s\bigr] = \mathbb{E}_{a\sim\pi(.|s_t)}\bigl[Q^{\pi}(s_t, a_t)\bigr] \\
              &= \sum_{a}\pi(a|s_t)*Q^{\pi}(s_t, a_t)
            \end{align*}
            \item For \colorbox{yellow}{Policy learning} , we assume $Q^{\pi}$ given, so we use neural network \colorbox{red}{$\pi(a|s_t; \theta)$} with weight $\theta$ to approximate $\pi(a|s_t)$\footnote{$\pi(a|s_t; \theta)\approx \pi(a|s_t)$} 

            \item For \colorbox{yellow}{Actor-Critic}, we approximate both $\pi$ and $Q^{\pi}$, with weight $\theta$, and $w$, by neural network \colorbox{red}{$\pi(a|s_t; \theta)$} and \colorbox{red}{$Q^{\pi}(s_t, a_t)$} 
        \end{itemize}
    \end{frame}
    
    \begin{frame}[fragile]{Actor Critic Algorithm : Pseudocode}
      % ---- White‑background box that can break across pages ----
      \begin{tcolorbox}
    
        \scriptsize  % <— just switch font, **no extra braces**
    
        \begin{algorithmic}[1]
          \Require policy network parameters $\theta$, Q-network parameters $w$, learning rates $\alpha_\theta,\alpha_w$, discount factor $\gamma$
          \Ensure updated parameters $\theta$, $w$
          \For{each episode}
            \State initialize state $s$
            \For{each step in the episode}
              \State Agent observe state $s_t$ and sample action $a_t \sim \pi_\theta(a \mid s)$
              \State execute $a_t$; observe reward $r$ and next state $s'$
              \State Randomly sample next action $a' \sim \pi_\theta(a' \mid s')$
              \State Evaluate value network $q_t = Q(s,a;w)$ and $q_{t+1} = Q(s',a';w)$
              \State Compute TD error : $\delta_t = q_t-(r_t+\gamma *q_{t+1})$\footnote{The theoretical basis of TD error is derived from bellman equation}
              \State $w \gets w + \alpha_w\,\delta_t\,\nabla_w Q(s,a;w)$  \Comment{Critic update}
              \State $\theta \gets \theta + \alpha_\theta\,\delta_t\,\nabla_\theta \log \pi_\theta(a \mid s)$\footnote{{The theoretical basis of actor update is based on policy gradient (i.e maximizing state-value function )}} \Comment{Actor update}
              \State $s \gets s'$
              \If{$s$ is terminal}
                \State \textbf{break}
              \EndIf
            \EndFor
          \EndFor
        \end{algorithmic}
    
      \end{tcolorbox}
    \end{frame}

    \begin{frame}[fragile]{From AC to SAC}
        \paragraph{1. 熵正則化後的目標函數}
        \begin{equation}
        J(\pi)=
        \mathbb{E}_{\tau\sim\pi}\Bigl[
          \sum_{t=0}^{\infty}\gamma^{t}\bigl(
            r(s_t,a_t)-\alpha\log\pi(a_t\mid s_t)
          \bigr)
        \Bigr].
        \end{equation}

        \paragraph{2. 定義軟 $Q$ 與軟 $V$}\footnote{參考\href{https://zhuanlan.zhihu.com/p/70360272}{知乎}以及論文\href{https://arxiv.org/pdf/1812.05905}{Soft Actor-Critic Algorithms and Applications}}
            \begin{align}
            Q^{\pi}(s,a)
              &=\mathbb{E}\!\Bigl[
                  \sum_{t=0}^{\infty}\gamma^{t}\tilde r(s_t,a_t,\pi)
                  \,\Bigl|\,s_0=s,a_0=a
                \Bigr],\\
            V^{\pi}(s)
              &=\mathbb{E}_{a\sim\pi}\bigl[
                  Q^{\pi}(s,a)-\alpha\log\pi(a\mid s)
                \bigr],
            \end{align}
            其中 $\tilde r(s,a,\pi)=r(s,a)-\alpha\log\pi(a\mid s)$。
    
    \end{frame}

    \begin{frame}[fragile]{Con'd}
        \paragraph{3. 軟 Bellman 期望方程}
            \begin{equation}
            \boxed{\,
              Q^{\pi}(s,a)=
              \tilde r(s,a,\pi)+
              \gamma\,
              \mathbb{E}_{s'\sim P}\bigl[V^{\pi}(s')\bigr]
            \,}.
            \end{equation}
            
            \paragraph{4. 最優值函數—soft-max 形式}
            \begin{equation}
            V^{*}(s)=
            \max_{\pi}V^{\pi}(s)=
            \begin{cases}
            \displaystyle
            \alpha\log\!\sum_{a}
              \exp\!\Bigl(\tfrac{1}{\alpha}Q^{*}(s,a)\Bigr),
            & \text{離散動作},\\[8pt]
            \displaystyle
            \alpha\log\!\int
              \exp\!\Bigl(\tfrac{1}{\alpha}Q^{*}(s,a)\Bigr)\,da,
            & \text{連續動作}.
            \end{cases}
            \end{equation}
    \end{frame}

    \begin{frame}[fragile]{Con'd}
        \paragraph{5. 軟 Bellman 最優方程}
            \begin{align}
            \boxed{
              Q^{*}(s,a)=
              r(s,a)+
              \gamma\,\mathbb{E}_{s'\sim P}\bigl[V^{*}(s')\bigr]
            },
            \label{eq:softQ} \\[6pt]
            \boxed{
              V^{*}(s)=
              \alpha\log\!\int
                \exp\!\bigl(\tfrac{1}{\alpha}Q^{*}(s,a)\bigr)\,da
            }.
            \label{eq:softV}
            \end{align}

        \paragraph{6. 小結:簡單來說就是加上了Maximum Entropy的設計,使設計可以幫助Agent 探索更多Polciy (輸出更分散的策略)}
    \end{frame}

    

    \begin{frame}[fragile]{Soft Actor Critic Algorithm : Pseudocode}
      \vspace{-2}
      \begin{tcolorbox}
        \tiny
        \begin{algorithmic}[1]
          \Require policy parameters $\theta$, twin--Q parameters $w_1,w_2$,
                  target Q parameters $\bar w_1,\bar w_2$,
                  entropy temperature $\alpha$, step sizes $\eta_\theta,\eta_w,\eta_\alpha$, discount $\gamma$
          \State initialize replay buffer $\mathcal{D}$
          \For{each episode}
            \State reset environment and get $s$
            \For{each step}
              \State sample $\tilde a_t = \tanh(\mu_\theta(s)+\sigma_\theta(s)\varepsilon)$, $\varepsilon\!\sim\!\mathcal{N}(0,I)$
              \State execute $\tilde a_t$; observe $(r,s')$; store $(s,\tilde a_t,r,s')$ in $\mathcal{D}$
              \For{each gradient update}
                \State sample mini-batch $\{(s_i,a_i,r_i,s'_i)\}$ from $\mathcal{D}$
                \State $a'_i\!\sim\!\pi_\theta(\cdot\mid s'_i)$ using reparameterization
                \State $y_i \gets r_i+\gamma\bigl[\min_{j=1,2}Q_{\bar w_j}(s'_i,a'_i)-\alpha\log\pi_\theta(a'_i\mid s'_i)\bigr]$
                \State $w_j \gets w_j -\eta_w\nabla_{w_j}\frac12\bigl(Q_{w_j}(s_i,a_i)-y_i\bigr)^2$ \Comment{Critic update}
                \State $\theta \gets \theta -\eta_\theta\nabla_\theta
                       \bigl[\alpha\log\pi_\theta(a_i\mid s_i)-\min_j Q_{w_j}(s_i,a_i)\bigr]$ \Comment{Actor update}
                \If{temperature tuning}
                  \State $\alpha\gets\alpha-\eta_\alpha\nabla_\alpha
                         \bigl[-\alpha\log\pi_\theta(a_i\mid s_i)-\alpha\mathcal{H}_{\text{target}}\bigr]$
                \EndIf
                \State $\bar w_j\gets\rho w_j + (1-\rho)\bar w_j$ \Comment{Soft target update}
              \EndFor
              \State $s\gets s'$
              \If{$s$ terminal} \textbf{break} \EndIf
            \EndFor
          \EndFor
        \end{algorithmic}
      \end{tcolorbox}
    \end{frame}

    

    

\section{Mapping RL and Economic}



\end{document}

