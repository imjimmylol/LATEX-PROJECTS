% !TEX program = xelatex
\documentclass{beamer}

% --------------------------------------------------
% Packages and Bibliography
% --------------------------------------------------
\usepackage{hyperref}
\usepackage{adjustbox}
\usepackage{float}
\floatplacement{algorithm}{H}
\usepackage{algorithm}
\usepackage{algorithmicx}
\usepackage{algcompatible}
\usepackage{algpseudocode}
\usepackage[normalem]{ulem}        % for underlining, strikeout

% Bibliography
\usepackage[backend=biber, style=numeric, sorting=none, doi=true, url=true]{biblatex}
\addbibresource{ref.bib}

% --------------------------------------------------
% XeLaTeX and CJK Fonts
% --------------------------------------------------
\usepackage[CJKmath=true, AutoFakeBold=true]{xeCJK}
\setCJKmainfont{AR PL KaitiM Big5}
\setCJKsansfont{AR PL KaitiM Big5}
\setCJKmonofont{AR PL KaitiM Big5}
\newCJKfontfamily\Kai{標楷體}
\XeTeXlinebreaklocale "zh"

% --------------------------------------------------
% Fonts, Colors, and Theme
% --------------------------------------------------
\usepackage[T1]{fontenc}
\usepackage{latexsym, amssymb, amsfonts, amsmath, amsthm, mathrsfs, mathptmx}
\usepackage{graphicx, pstricks}
\usepackage{tikz}
\usetikzlibrary{arrows.meta, positioning}
\usepackage{multicol}
\usepackage{booktabs}
\usepackage{calligra}
\usepackage{soul}
\usepackage{enumitem}
\usefonttheme[onlymath]{serif}

% Code listings
\usepackage{listings}
\lstset{
    language=[LaTeX]TeX,
    basicstyle=\ttfamily\footnotesize,
    keywordstyle=\bfseries\color[RGB]{0,0,139},
    stringstyle=\color[RGB]{50,50,50},
    numbers=left,
    numberstyle=\small\color{gray},
    rulesepcolor=\color{red!20!green!20!blue!20},
    frame=shadowbox,
}

% --------------------------------------------------
% Itemize and Enumerate Customization
% --------------------------------------------------
% Beamer item symbols
\setbeamertemplate{itemize item}{$\bullet$}
\setbeamertemplate{itemize subitem}{--}
\setbeamertemplate{itemize subsubitem}{$\circ$}

% enumitem settings
\setlist[itemize,1]{label=\usebeamertemplate{itemize item}, leftmargin=1.5em, itemsep=0.3em}
\setlist[itemize,2]{label=--, leftmargin=2.5em, labelsep=0.5em, topsep=0.3em, itemsep=0.1em}
\setlist[itemize,3]{label=$\circ$, leftmargin=3em, labelsep=0.5em, topsep=0.3em, itemsep=0.1em}
\setlist[enumerate,1]{label=\arabic*., ref=\arabic*}
\setlist[enumerate,2]{label=\alph*), ref=\theenumi\alph*}

% --------------------------------------------------
% Custom Commands and Metadata
% --------------------------------------------------
\renewcommand{\today}{\number\year 年\number\month 月\number\day 日}
\renewcommand{\alert}[1]{\textbf{\color{swufe}#1}}

% University style
\usepackage{JNU}

% Presentation metadata
\author{陳景龢}
\title{Paper Detail}
\subtitle{TAX-AI}
\institute[NCCU]{\normalsize Econ Dept\quad NCCU}
\date{2025年6月3日}

% --------------------------------------------------
% Document
% --------------------------------------------------
\begin{document}

\begin{frame}
  \titlepage
  \begin{figure}[htpb]
    \centering
    % 如果有校徽或標誌,請放在 pic/jiangnan_logo.png
    % \includegraphics[width=0.55\linewidth]{pic/jiangnan_logo.png}
  \end{figure}
\end{frame}

\begin{frame}
  \tableofcontents[sectionstyle=show,subsectionstyle=show/shaded/hide,subsubsectionstyle=show/shaded/hide]
\end{frame}

\section{Introduction}
\begin{frame}{Background : Why MARL ?}
  \begin{itemize}
    \item Extracting relevant and actionable information from a complex society is arduous
    \item Modeling a vast and heterogeneous population
    \item Response of individuals to incentives remains highly unpredictable
    \item Traditional ABM suffers from simplicity and subjectivity in setting parameters and behavior rules
  \end{itemize}
\end{frame}

\begin{frame}{Background : The benefits of MARL}
  \begin{itemize}
    \item Offering optimal actions based on evolving state information
    \item MARL algorithms perform well in partially observable environments and adaptively learn equilibrium solutions
  \end{itemize}
\end{frame}

\section{Recap}
\begin{frame}{Existing Literature}
  The existing literature--AI Economist \cite{zheng2022ai} and RBC model \cite{curry2022analyzing}--exhibit limitations:
  \begin{itemize}
    \item Partial grounding in economic theory
    \item Limited scalability in simulating many agents
    \item Absence of calibration using real-world data
  \end{itemize}
\end{frame}

\section{Model Setup}
\begin{frame}{Economic Backbone:model Bewley-Aiyagari}
  \begin{itemize}
    \item Widely used to study:
    \begin{itemize}
      \item Capital market frictions, wealth distribution, economic growth
    \end{itemize}
    \item Four key roles:
    \begin{itemize}
      \item $N$ households, representative firm, financial intermediary, government
    \end{itemize}
  \end{itemize}
\end{frame}

\section{Simulation Detail}
% 在此添加模擬細節

\section{Learning Algorithm}
% 在此添加學習算法內容

\begin{frame}[allowframebreaks]{References}
    \nocite{*}
    \printbibliography
\end{frame}

\end{document}
